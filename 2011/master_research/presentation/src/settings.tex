

% * * * * * * * * * * * * * * PACKAGES CLASSIQUES * * * * * * * * * * * * * * *
\usepackage[T1]{fontenc} 
\usepackage[utf8]{inputenc}
\usepackage[frenchb]{babel}
\usepackage{amsmath}
\usepackage{xcolor}  
\usepackage{graphicx}
\usepackage{tikz}
\usepackage{xmpmulti}
%\usepackage{bibentry}										% Bibentry in footnote for example
%\usepackage{inlinebib}
%\nobibliography*
%\usepackage{natbib}										% Reimplement the LATEX \cite command

% * * * * * * * * * * * * * * *  ANIMATIONS * * * * * * * * * * * * * * * * * *
\usepackage{animate}

% * * * * * * * * * * * * * * CHOIX DU THEME  * * * * * * * * * * * * * * * * *
\usepackage{beamerthemeFrankfurt}                                % un theme voir .../beamer/theme/

% * * * * * * * * * * * * * LA BARRE DE NAVIGATION  * * * * * * * * * * * * * *
% commenter la ligne pour supprimer un éléments
\setbeamertemplate{navigation symbols}{%
%	\insertslidenavigationsymbol%
%	\insertframenavigationsymbol%
%	\insertsubsectionnavigationsymbol%
%	\insertsectionnavigationsymbol%
%	\insertdocnavigationsymbol%
%	\insertbackfindforwardnavigationsymbol%
}

% * * * * * * * * * * * * * * * * * TEXTPOS * * * * * * * * * * * * * * * * * *
\usepackage[absolute,showboxes,overlay]{textpos}
\TPshowboxestrue                                              % affiche le contour des textblocks
\TPshowboxesfalse                                             % fait disparaitre le contour des textblocks
\textblockorigin{2mm}{8mm}                                    % origine des positions pour placer les textblocks

% * * * * * * * * * * * * * * * * * PICTURE * * * * * * * * * * * * * * * * * *
\usepackage{picture}
\setlength{\unitlength}{1mm}                                  % définition de l'unité

% * * * * * * * * * * * * * * *  LES BLOCKS * * * * * * * * * * * * * * * * * *
\setbeamertemplate{blocks}[rounded][shadow=true]              % pour des blocks arrondis
\setbeamercolor{block body alerted}{fg=white,bg=monred}       % ecrit en blanc sur fond rouge
\setbeamercolor{block body}{fg=white,bg=monbleu}              % ecrit en blanc sur fond bleu

% * * * * * * * * * * * * * DETAILS DE STYLE  * * * * * * * * * * * * * * * * *
\beamertemplatetransparentcovered                             % Fait afficher l'ensemble du frame en peu lisible (gris clair) dès l'ouverture

\setbeamertemplate{itemize item}[ball]                        % style item
\setbeamertemplate{itemize subitem}[triangle]                 % style subitem
\setbeamertemplate{footline}[page number]
%\setbeamertemplate{footline}[text line]{\href{http://www.christophe-rigaud.com}{Christophe Rigaud}}
  %\vspace*{-1cm}\centering\normalsize Some information\\[0.3em] some more info\\[0.3em]} 
  
%\logo{\includegraphics[height=5mm]{image/lisa.png}}           % définition du logo 

\renewcommand{\arraystretch}{1.4}                             % espacement des cellules du tableau 

\definecolor{monred}{HTML}{9D0909}                            % un rouge
\definecolor{monbleu}{HTML}{000066}                           % un bleu
\definecolor{monvert}{HTML}{00AE00}                           % un vert

%\renewcommand{\footnoterule}{}                                % supprime le trait au dessus des footnotes
%\renewcommand{\thefootnote}{\alph{footnote}}                  % numérotation par des lettres

% * * * * * * * * * * * * * *  PAGES DE TITRE * * * * * * * * * * * * * * * * *
\title[Abbrev. Title\hspace{2em}\insertframenumber/\inserttotalframenumber]{Image interpretation and conceptual graph integrating topologic and photometric knowledge}
%\subtitle{Application to synthetic image thresholding and medical image windowing}
\author[CR]{\href{http://www.christophe-rigaud.com}{Christophe RIGAUD}}
\institute{LABORATOIRE D’INGÉNIERIE DES SYSTÈMES AUTOMATISÉS}
\date{ 7 July 2011 }

% * * * * * * * * * * * * * * * *  SOMMAIRE * * * * * * * * * * * * * * * * *
\AtBeginSection[]{
  \begin{frame}{Road map}
  \small \tableofcontents[currentsection, hideothersubsections]
  \end{frame} 
}
% * * * * * * * * * * * * * PARAMETRES POUR PDF * * * * * * * * * * * * * * * *
\hypersetup{% Modifiez la valeur des champs suivants
	pdfauthor   = {Christophe Rigaud},%
	pdftitle    = {Titre},%
	pdfsubject  = {Sujet},%
	pdfkeywords = {Mots clés},%
	pdfcreator  = {PDFLaTeX},%
	pdfproducer = {PDFLaTeX},%
	%pdfpagemode = {FullScreen}%                           % ouvre le pdf directement en plein écran
}

